% Generated by Sphinx.
\def\sphinxdocclass{report}
\documentclass[a4paper,10pt,english]{sphinxmanual}
\usepackage[utf8]{inputenc}
\DeclareUnicodeCharacter{00A0}{\nobreakspace}
\usepackage{cmap}
\usepackage[T1]{fontenc}
\usepackage{amsfonts}
\usepackage{babel}
\usepackage{times}
\usepackage[Bjarne]{fncychap}
\usepackage{longtable}
\usepackage{sphinx}
\usepackage{multirow}
\usepackage{eqparbox}


\addto\captionsenglish{\renewcommand{\figurename}{Fig. }}
\addto\captionsenglish{\renewcommand{\tablename}{Table }}
\SetupFloatingEnvironment{literal-block}{name=Listing }



\title{biobeam Documentation}
\date{June 30, 2016}
\release{0.1.0}
\author{Martin Weigert}
\newcommand{\sphinxlogo}{\includegraphics{logo_biobeam_transp.png}\par}
\renewcommand{\releasename}{Release}
\setcounter{tocdepth}{1}
\makeindex

\makeatletter
\def\PYG@reset{\let\PYG@it=\relax \let\PYG@bf=\relax%
    \let\PYG@ul=\relax \let\PYG@tc=\relax%
    \let\PYG@bc=\relax \let\PYG@ff=\relax}
\def\PYG@tok#1{\csname PYG@tok@#1\endcsname}
\def\PYG@toks#1+{\ifx\relax#1\empty\else%
    \PYG@tok{#1}\expandafter\PYG@toks\fi}
\def\PYG@do#1{\PYG@bc{\PYG@tc{\PYG@ul{%
    \PYG@it{\PYG@bf{\PYG@ff{#1}}}}}}}
\def\PYG#1#2{\PYG@reset\PYG@toks#1+\relax+\PYG@do{#2}}

\expandafter\def\csname PYG@tok@gd\endcsname{\def\PYG@tc##1{\textcolor[rgb]{0.63,0.00,0.00}{##1}}}
\expandafter\def\csname PYG@tok@gu\endcsname{\let\PYG@bf=\textbf\def\PYG@tc##1{\textcolor[rgb]{0.50,0.00,0.50}{##1}}}
\expandafter\def\csname PYG@tok@gt\endcsname{\def\PYG@tc##1{\textcolor[rgb]{0.00,0.27,0.87}{##1}}}
\expandafter\def\csname PYG@tok@gs\endcsname{\let\PYG@bf=\textbf}
\expandafter\def\csname PYG@tok@gr\endcsname{\def\PYG@tc##1{\textcolor[rgb]{1.00,0.00,0.00}{##1}}}
\expandafter\def\csname PYG@tok@cm\endcsname{\let\PYG@it=\textit\def\PYG@tc##1{\textcolor[rgb]{0.25,0.50,0.56}{##1}}}
\expandafter\def\csname PYG@tok@vg\endcsname{\def\PYG@tc##1{\textcolor[rgb]{0.73,0.38,0.84}{##1}}}
\expandafter\def\csname PYG@tok@vi\endcsname{\def\PYG@tc##1{\textcolor[rgb]{0.73,0.38,0.84}{##1}}}
\expandafter\def\csname PYG@tok@mh\endcsname{\def\PYG@tc##1{\textcolor[rgb]{0.13,0.50,0.31}{##1}}}
\expandafter\def\csname PYG@tok@cs\endcsname{\def\PYG@tc##1{\textcolor[rgb]{0.25,0.50,0.56}{##1}}\def\PYG@bc##1{\setlength{\fboxsep}{0pt}\colorbox[rgb]{1.00,0.94,0.94}{\strut ##1}}}
\expandafter\def\csname PYG@tok@ge\endcsname{\let\PYG@it=\textit}
\expandafter\def\csname PYG@tok@vc\endcsname{\def\PYG@tc##1{\textcolor[rgb]{0.73,0.38,0.84}{##1}}}
\expandafter\def\csname PYG@tok@il\endcsname{\def\PYG@tc##1{\textcolor[rgb]{0.13,0.50,0.31}{##1}}}
\expandafter\def\csname PYG@tok@go\endcsname{\def\PYG@tc##1{\textcolor[rgb]{0.20,0.20,0.20}{##1}}}
\expandafter\def\csname PYG@tok@cp\endcsname{\def\PYG@tc##1{\textcolor[rgb]{0.00,0.44,0.13}{##1}}}
\expandafter\def\csname PYG@tok@gi\endcsname{\def\PYG@tc##1{\textcolor[rgb]{0.00,0.63,0.00}{##1}}}
\expandafter\def\csname PYG@tok@gh\endcsname{\let\PYG@bf=\textbf\def\PYG@tc##1{\textcolor[rgb]{0.00,0.00,0.50}{##1}}}
\expandafter\def\csname PYG@tok@ni\endcsname{\let\PYG@bf=\textbf\def\PYG@tc##1{\textcolor[rgb]{0.84,0.33,0.22}{##1}}}
\expandafter\def\csname PYG@tok@nl\endcsname{\let\PYG@bf=\textbf\def\PYG@tc##1{\textcolor[rgb]{0.00,0.13,0.44}{##1}}}
\expandafter\def\csname PYG@tok@nn\endcsname{\let\PYG@bf=\textbf\def\PYG@tc##1{\textcolor[rgb]{0.05,0.52,0.71}{##1}}}
\expandafter\def\csname PYG@tok@no\endcsname{\def\PYG@tc##1{\textcolor[rgb]{0.38,0.68,0.84}{##1}}}
\expandafter\def\csname PYG@tok@na\endcsname{\def\PYG@tc##1{\textcolor[rgb]{0.25,0.44,0.63}{##1}}}
\expandafter\def\csname PYG@tok@nb\endcsname{\def\PYG@tc##1{\textcolor[rgb]{0.00,0.44,0.13}{##1}}}
\expandafter\def\csname PYG@tok@nc\endcsname{\let\PYG@bf=\textbf\def\PYG@tc##1{\textcolor[rgb]{0.05,0.52,0.71}{##1}}}
\expandafter\def\csname PYG@tok@nd\endcsname{\let\PYG@bf=\textbf\def\PYG@tc##1{\textcolor[rgb]{0.33,0.33,0.33}{##1}}}
\expandafter\def\csname PYG@tok@ne\endcsname{\def\PYG@tc##1{\textcolor[rgb]{0.00,0.44,0.13}{##1}}}
\expandafter\def\csname PYG@tok@nf\endcsname{\def\PYG@tc##1{\textcolor[rgb]{0.02,0.16,0.49}{##1}}}
\expandafter\def\csname PYG@tok@si\endcsname{\let\PYG@it=\textit\def\PYG@tc##1{\textcolor[rgb]{0.44,0.63,0.82}{##1}}}
\expandafter\def\csname PYG@tok@s2\endcsname{\def\PYG@tc##1{\textcolor[rgb]{0.25,0.44,0.63}{##1}}}
\expandafter\def\csname PYG@tok@nt\endcsname{\let\PYG@bf=\textbf\def\PYG@tc##1{\textcolor[rgb]{0.02,0.16,0.45}{##1}}}
\expandafter\def\csname PYG@tok@nv\endcsname{\def\PYG@tc##1{\textcolor[rgb]{0.73,0.38,0.84}{##1}}}
\expandafter\def\csname PYG@tok@s1\endcsname{\def\PYG@tc##1{\textcolor[rgb]{0.25,0.44,0.63}{##1}}}
\expandafter\def\csname PYG@tok@ch\endcsname{\let\PYG@it=\textit\def\PYG@tc##1{\textcolor[rgb]{0.25,0.50,0.56}{##1}}}
\expandafter\def\csname PYG@tok@m\endcsname{\def\PYG@tc##1{\textcolor[rgb]{0.13,0.50,0.31}{##1}}}
\expandafter\def\csname PYG@tok@gp\endcsname{\let\PYG@bf=\textbf\def\PYG@tc##1{\textcolor[rgb]{0.78,0.36,0.04}{##1}}}
\expandafter\def\csname PYG@tok@sh\endcsname{\def\PYG@tc##1{\textcolor[rgb]{0.25,0.44,0.63}{##1}}}
\expandafter\def\csname PYG@tok@ow\endcsname{\let\PYG@bf=\textbf\def\PYG@tc##1{\textcolor[rgb]{0.00,0.44,0.13}{##1}}}
\expandafter\def\csname PYG@tok@sx\endcsname{\def\PYG@tc##1{\textcolor[rgb]{0.78,0.36,0.04}{##1}}}
\expandafter\def\csname PYG@tok@bp\endcsname{\def\PYG@tc##1{\textcolor[rgb]{0.00,0.44,0.13}{##1}}}
\expandafter\def\csname PYG@tok@c1\endcsname{\let\PYG@it=\textit\def\PYG@tc##1{\textcolor[rgb]{0.25,0.50,0.56}{##1}}}
\expandafter\def\csname PYG@tok@o\endcsname{\def\PYG@tc##1{\textcolor[rgb]{0.40,0.40,0.40}{##1}}}
\expandafter\def\csname PYG@tok@kc\endcsname{\let\PYG@bf=\textbf\def\PYG@tc##1{\textcolor[rgb]{0.00,0.44,0.13}{##1}}}
\expandafter\def\csname PYG@tok@c\endcsname{\let\PYG@it=\textit\def\PYG@tc##1{\textcolor[rgb]{0.25,0.50,0.56}{##1}}}
\expandafter\def\csname PYG@tok@mf\endcsname{\def\PYG@tc##1{\textcolor[rgb]{0.13,0.50,0.31}{##1}}}
\expandafter\def\csname PYG@tok@err\endcsname{\def\PYG@bc##1{\setlength{\fboxsep}{0pt}\fcolorbox[rgb]{1.00,0.00,0.00}{1,1,1}{\strut ##1}}}
\expandafter\def\csname PYG@tok@mb\endcsname{\def\PYG@tc##1{\textcolor[rgb]{0.13,0.50,0.31}{##1}}}
\expandafter\def\csname PYG@tok@ss\endcsname{\def\PYG@tc##1{\textcolor[rgb]{0.32,0.47,0.09}{##1}}}
\expandafter\def\csname PYG@tok@sr\endcsname{\def\PYG@tc##1{\textcolor[rgb]{0.14,0.33,0.53}{##1}}}
\expandafter\def\csname PYG@tok@mo\endcsname{\def\PYG@tc##1{\textcolor[rgb]{0.13,0.50,0.31}{##1}}}
\expandafter\def\csname PYG@tok@kd\endcsname{\let\PYG@bf=\textbf\def\PYG@tc##1{\textcolor[rgb]{0.00,0.44,0.13}{##1}}}
\expandafter\def\csname PYG@tok@mi\endcsname{\def\PYG@tc##1{\textcolor[rgb]{0.13,0.50,0.31}{##1}}}
\expandafter\def\csname PYG@tok@kn\endcsname{\let\PYG@bf=\textbf\def\PYG@tc##1{\textcolor[rgb]{0.00,0.44,0.13}{##1}}}
\expandafter\def\csname PYG@tok@cpf\endcsname{\let\PYG@it=\textit\def\PYG@tc##1{\textcolor[rgb]{0.25,0.50,0.56}{##1}}}
\expandafter\def\csname PYG@tok@kr\endcsname{\let\PYG@bf=\textbf\def\PYG@tc##1{\textcolor[rgb]{0.00,0.44,0.13}{##1}}}
\expandafter\def\csname PYG@tok@s\endcsname{\def\PYG@tc##1{\textcolor[rgb]{0.25,0.44,0.63}{##1}}}
\expandafter\def\csname PYG@tok@kp\endcsname{\def\PYG@tc##1{\textcolor[rgb]{0.00,0.44,0.13}{##1}}}
\expandafter\def\csname PYG@tok@w\endcsname{\def\PYG@tc##1{\textcolor[rgb]{0.73,0.73,0.73}{##1}}}
\expandafter\def\csname PYG@tok@kt\endcsname{\def\PYG@tc##1{\textcolor[rgb]{0.56,0.13,0.00}{##1}}}
\expandafter\def\csname PYG@tok@sc\endcsname{\def\PYG@tc##1{\textcolor[rgb]{0.25,0.44,0.63}{##1}}}
\expandafter\def\csname PYG@tok@sb\endcsname{\def\PYG@tc##1{\textcolor[rgb]{0.25,0.44,0.63}{##1}}}
\expandafter\def\csname PYG@tok@k\endcsname{\let\PYG@bf=\textbf\def\PYG@tc##1{\textcolor[rgb]{0.00,0.44,0.13}{##1}}}
\expandafter\def\csname PYG@tok@se\endcsname{\let\PYG@bf=\textbf\def\PYG@tc##1{\textcolor[rgb]{0.25,0.44,0.63}{##1}}}
\expandafter\def\csname PYG@tok@sd\endcsname{\let\PYG@it=\textit\def\PYG@tc##1{\textcolor[rgb]{0.25,0.44,0.63}{##1}}}

\def\PYGZbs{\char`\\}
\def\PYGZus{\char`\_}
\def\PYGZob{\char`\{}
\def\PYGZcb{\char`\}}
\def\PYGZca{\char`\^}
\def\PYGZam{\char`\&}
\def\PYGZlt{\char`\<}
\def\PYGZgt{\char`\>}
\def\PYGZsh{\char`\#}
\def\PYGZpc{\char`\%}
\def\PYGZdl{\char`\$}
\def\PYGZhy{\char`\-}
\def\PYGZsq{\char`\'}
\def\PYGZdq{\char`\"}
\def\PYGZti{\char`\~}
% for compatibility with earlier versions
\def\PYGZat{@}
\def\PYGZlb{[}
\def\PYGZrb{]}
\makeatother

\renewcommand\PYGZsq{\textquotesingle}

\begin{document}

\maketitle
\tableofcontents
\phantomsection\label{index::doc}


Biobeams is awesome. And so are you!

This is something I want to say that is not in the docstring.


\chapter{Introduction}
\label{intro:introduction}\label{intro:welcome-to-biobeams}\label{intro::doc}
Biobeams is awesome. And so are you!

This is something I want to say that is not in the docstring.


\chapter{Installation}
\label{installing:installation}\label{installing::doc}
To nicely render the 3d output it is advisible to install \emph{Spimagine}, an OpenCL accelerated renderer \href{https://github.com/maweigert/spimagine}{Spimagine}

\begin{Verbatim}[commandchars=\\\{\}]
pip install spimagine
\end{Verbatim}

After that you should be able to simple do

\begin{Verbatim}[commandchars=\\\{\}]
pip install biobeams
\end{Verbatim}


\chapter{Basic Usage}
\label{basic::doc}\label{basic:basic-usage}\setbox0\vbox{
\begin{minipage}{0.95\linewidth}
\textbf{Contents}

\medskip

\begin{itemize}
\item {} 
\phantomsection\label{basic:id1}{\hyperref[basic:basic\string-usage]{\emph{Basic Usage}}}
\begin{itemize}
\item {} 
\phantomsection\label{basic:id2}{\hyperref[basic:beam\string-propagation]{\emph{Beam propagation}}}

\end{itemize}

\end{itemize}
\end{minipage}}
\begin{center}\setlength{\fboxsep}{5pt}\shadowbox{\box0}\end{center}


\section{Beam propagation}
\label{basic:beam-propagation}\index{Bpm3d (class in biobeam)}

\begin{fulllineitems}
\phantomsection\label{basic:biobeam.Bpm3d}\pysiglinewithargsret{\strong{class }\code{biobeam.}\bfcode{Bpm3d}}{\emph{size=None}, \emph{shape=None}, \emph{units=None}, \emph{dn=None}, \emph{lam=0.5}, \emph{n0=1.0}, \emph{simul\_xy=None}, \emph{simul\_z=1}, \emph{n\_volumes=1}, \emph{enforce\_subsampled=False}, \emph{fftplan\_kwargs=\{\}}}{}
the main class for gpu accelerated bpm propagation
\index{\_\_init\_\_() (biobeam.Bpm3d method)}

\begin{fulllineitems}
\phantomsection\label{basic:biobeam.Bpm3d.__init__}\pysiglinewithargsret{\bfcode{\_\_init\_\_}}{\emph{size=None}, \emph{shape=None}, \emph{units=None}, \emph{dn=None}, \emph{lam=0.5}, \emph{n0=1.0}, \emph{simul\_xy=None}, \emph{simul\_z=1}, \emph{n\_volumes=1}, \emph{enforce\_subsampled=False}, \emph{fftplan\_kwargs=\{\}}}{}~\begin{quote}\begin{description}
\item[{Parameters}] \leavevmode\begin{itemize}
\item {} 
\textbf{\texttt{size}} (\emph{\texttt{(Sx,Sy,Sz)}}) -- the size of the geometry in microns (Sx,Sy,Sz)

\item {} 
\textbf{\texttt{shape}} (\emph{\texttt{(Nx,Ny,Nz)}}) -- the shape of the geometry in pixels (Nx,Ny,Nz)

\item {} 
\textbf{\texttt{units}} (\emph{\texttt{(dx,dy,dz)}}) -- the voxelsizes in microns (dx,dy,dz)

\item {} 
\textbf{\texttt{dn}} (\emph{\texttt{ndarray (float32\textbar{}complex64)}}) -- refractive index distribution, dn.shape != (Nz,Ny,Nx)

\item {} 
\textbf{\texttt{lam}} (\emph{\texttt{float}}) -- the wavelength in microns

\item {} 
\textbf{\texttt{n0}} (\emph{\texttt{float}}) -- the refractive index of the surrounding media

\item {} 
\textbf{\texttt{simul\_xy}} (\emph{\texttt{(Nx,Ny,Nz), optional}}) -- the shape of the 2d computational geometry in pixels (Nx,Ny)
(e.g. subsampling in xy)

\item {} 
\textbf{\texttt{simul\_z}} (\emph{\texttt{int, optional}}) -- the subsampling factor along z

\item {} 
\textbf{\texttt{n\_volumes}} (\emph{\texttt{int}}) -- splits the domain into chunks if GPU memory is not
large enough (will be set automatically)

\end{itemize}

\end{description}\end{quote}
\paragraph{Example}

\begin{Verbatim}[commandchars=\\\{\}]
\PYG{g+gp}{\PYGZgt{}\PYGZgt{}\PYGZgt{} }\PYG{n}{m} \PYG{o}{=} \PYG{n}{Bpm3d}\PYG{p}{(}\PYG{n}{size} \PYG{o}{=} \PYG{p}{(}\PYG{l+m+mi}{10}\PYG{p}{,}\PYG{l+m+mi}{10}\PYG{p}{,}\PYG{l+m+mi}{10}\PYG{p}{)}\PYG{p}{,}\PYG{n}{shape} \PYG{o}{=} \PYG{p}{(}\PYG{l+m+mi}{256}\PYG{p}{,}\PYG{l+m+mi}{256}\PYG{p}{,}\PYG{l+m+mi}{256}\PYG{p}{)}\PYG{p}{,}\PYG{n}{units} \PYG{o}{=} \PYG{p}{(}\PYG{l+m+mf}{0.1}\PYG{p}{,}\PYG{l+m+mf}{0.1}\PYG{p}{,}\PYG{l+m+mf}{0.1}\PYG{p}{)}\PYG{p}{,}\PYG{n}{lam} \PYG{o}{=} \PYG{l+m+mf}{0.488}\PYG{p}{,}\PYG{n}{n0} \PYG{o}{=} \PYG{l+m+mf}{1.33}\PYG{p}{)}
\end{Verbatim}

\end{fulllineitems}

\index{aberr\_at() (biobeam.Bpm3d method)}

\begin{fulllineitems}
\phantomsection\label{basic:biobeam.Bpm3d.aberr_at}\pysiglinewithargsret{\bfcode{aberr\_at}}{\emph{NA=0.4}, \emph{center=(0}, \emph{0}, \emph{0)}, \emph{n\_zern=20}, \emph{n\_integration\_steps=200}}{}
c = (cx,cy,cz) in realtive pixel coordinates wrt the center

returns phi, zern

\end{fulllineitems}

\index{aberr\_field\_grid() (biobeam.Bpm3d method)}

\begin{fulllineitems}
\phantomsection\label{basic:biobeam.Bpm3d.aberr_field_grid}\pysiglinewithargsret{\bfcode{aberr\_field\_grid}}{\emph{NA}, \emph{cxs}, \emph{cys}, \emph{cz}, \emph{n\_zern=20}, \emph{n\_integration\_steps=200}}{}
cxs, cys are equally spaced 1d arrays defining the grid

\end{fulllineitems}


\end{fulllineitems}



\chapter{Input Beam patterns}
\label{beams:input-beam-patterns}\label{beams::doc}\setbox0\vbox{
\begin{minipage}{0.95\linewidth}
\textbf{Contents}

\medskip

\begin{itemize}
\item {} 
\phantomsection\label{beams:id7}{\hyperref[beams:input\string-beam\string-patterns]{\emph{Input Beam patterns}}}
\begin{itemize}
\item {} 
\phantomsection\label{beams:id8}{\hyperref[beams:gaussian\string-bessel\string-beams]{\emph{Gaussian/Bessel beams}}}

\item {} 
\phantomsection\label{beams:id9}{\hyperref[beams:cylindrical\string-lens]{\emph{Cylindrical Lens}}}

\item {} 
\phantomsection\label{beams:id10}{\hyperref[beams:bessel\string-lattices]{\emph{Bessel Lattices}}}

\end{itemize}

\end{itemize}
\end{minipage}}
\begin{center}\setlength{\fboxsep}{5pt}\shadowbox{\box0}\end{center}


\section{Gaussian/Bessel beams}
\label{beams:gaussian-bessel-beams}
\textbf{Gaussian/Bessel beams}

{\includegraphics{{pupil_gauss}.png}\hspace*{\fill}}

{\hspace*{\fill}\includegraphics{{pupil_bessel}.png}}
\index{focus\_field\_beam() (in module biobeam)}

\begin{fulllineitems}
\phantomsection\label{beams:biobeam.focus_field_beam}\pysiglinewithargsret{\code{biobeam.}\bfcode{focus\_field\_beam}}{\emph{shape}, \emph{units}, \emph{lam=0.5}, \emph{NA=0.6}, \emph{n0=1.0}, \emph{return\_all\_fields=False}, \emph{n\_integration\_steps=200}}{}
calculates the focus field for a perfect, aberration free optical system for
x polzarized illumination via the vectorial debye diffraction integral (see \footnote[1]{
Matthew R. Foreman, Peter Toeroek, \emph{Computational methods in vectorial imaging}, Journal of Modern Optics, 2011, 58, 5-6, 339
}).
The pupil function is given by numerical aperture(s) NA (that can be a list to
model bessel beams, see further below)
\begin{quote}\begin{description}
\item[{Parameters}] \leavevmode\begin{itemize}
\item {} 
\textbf{\texttt{shape}} (\emph{\texttt{Nx,Ny,Nz}}) -- the shape of the geometry

\item {} 
\textbf{\texttt{units}} (\emph{\texttt{dx,dy,dz}}) -- the pixel sizes in microns

\item {} 
\textbf{\texttt{lam}} (\emph{\texttt{float}}) -- the wavelength of light used in microns

\item {} 
\textbf{\texttt{NA}} (\emph{\texttt{float/list}}) -- the numerical aperture(s) of the illumination objective
that is either a single number (for gaussian beams) or an
even length list of NAs (for bessel beams), e.g.
\emph{NA = {[}0.5,0.55{]}} lets light through the annulus 0.5\textless{}0.55 (making a bessel beam ) or
\emph{NA = {[}0.1,0.2,0.5,0.6{]}} lets light through the annulus 0.1\textless{}0.2 and 0.5\textless{}0.6 making a
beating double bessel beam...

\item {} 
\textbf{\texttt{n0}} (\emph{\texttt{float}}) -- the refractive index of the medium

\item {} 
\textbf{\texttt{n\_integration\_steps}} (\emph{\texttt{int}}) -- number of integration steps to perform

\item {} 
\textbf{\texttt{return\_all\_fields}} (\emph{\texttt{boolean}}) -- if True returns also the complex vectorial field components

\end{itemize}

\item[{Returns}] \leavevmode
\begin{itemize}
\item {} 
\textbf{u} (\emph{ndarray}) --
the intensity of the focus field

\item {} 
\textbf{(u,ex,ey,ez)} (\emph{list(ndarray)}) --
the intensity of the focus field and the complex field components (if return\_all\_fields is True)

\end{itemize}


\end{description}\end{quote}
\paragraph{Example}

\begin{Verbatim}[commandchars=\\\{\}]
\PYG{g+gp}{\PYGZgt{}\PYGZgt{}\PYGZgt{} }\PYG{n}{u} \PYG{o}{=} \PYG{n}{focus\PYGZus{}field\PYGZus{}beam}\PYG{p}{(}\PYG{p}{(}\PYG{l+m+mi}{128}\PYG{p}{,}\PYG{l+m+mi}{128}\PYG{p}{,}\PYG{l+m+mi}{128}\PYG{p}{)}\PYG{p}{,} \PYG{p}{(}\PYG{l+m+mf}{0.1}\PYG{p}{,}\PYG{l+m+mf}{0.1}\PYG{p}{,}\PYG{o}{.}\PYG{l+m+mi}{1}\PYG{p}{)}\PYG{p}{,} \PYG{n}{lam}\PYG{o}{=}\PYG{o}{.}\PYG{l+m+mi}{5}\PYG{p}{,} \PYG{n}{NA} \PYG{o}{=} \PYG{o}{.}\PYG{l+m+mi}{4}\PYG{p}{)}
\end{Verbatim}
\paragraph{References}

\end{fulllineitems}

\index{focus\_field\_beam\_plane() (in module biobeam)}

\begin{fulllineitems}
\phantomsection\label{beams:biobeam.focus_field_beam_plane}\pysiglinewithargsret{\code{biobeam.}\bfcode{focus\_field\_beam\_plane}}{\emph{shape=(128}, \emph{128)}, \emph{units=(0.1}, \emph{0.1)}, \emph{z=0.0}, \emph{lam=0.5}, \emph{NA=0.6}, \emph{n0=1.0}, \emph{ex\_g=None}, \emph{n\_integration\_steps=200}}{}
calculates the complex 2d input field at position -z of a      perfect, aberration free optical system
\begin{quote}\begin{description}
\item[{Parameters}] \leavevmode\begin{itemize}
\item {} 
\textbf{\texttt{shape}} (\emph{\texttt{Nx,Ny}}) -- the 2d shape of the geometry

\item {} 
\textbf{\texttt{units}} (\emph{\texttt{dx,dy}}) -- the pixel sizes in microns

\item {} 
\textbf{\texttt{z}} (\emph{\texttt{float}}) -- defocus position in microns, such that the beam would focus at z
e.g. an input field with z = 10. would hav its focus spot after 10 microns

\item {} 
\textbf{\texttt{lam}} (\emph{\texttt{float}}) -- the wavelength of light used in microns

\item {} 
\textbf{\texttt{NA}} (\emph{\texttt{float/list}}) -- the numerical aperture(s) of the illumination objective
that is either a single number (for gaussian beams) or an
even length list of NAs (for bessel beams), e.g.
NA = {[}0.5,0.55{]} lets light through the annulus 0.5\textless{}0.55 (making a bessel beam ) or
NA = {[}0.1,0.2,0.5,0.6{]} lets light through the annulus 0.1\textless{}0.2 and 0.5\textless{}0.6 making a
beating double bessel beam...

\item {} 
\textbf{\texttt{n0}} (\emph{\texttt{float}}) -- the refractive index of the medium

\item {} 
\textbf{\texttt{n\_integration\_steps}} (\emph{\texttt{int}}) -- number of integration steps to perform

\end{itemize}

\item[{Returns}] \leavevmode
\textbf{ex} --
the complex field

\item[{Return type}] \leavevmode
ndarray

\end{description}\end{quote}
\paragraph{Example}

\begin{Verbatim}[commandchars=\\\{\}]
\PYG{g+gp}{\PYGZgt{}\PYGZgt{}\PYGZgt{} }\PYG{c+c1}{\PYGZsh{} the input pattern of a bessel beam that will focus after 4 microns}
\PYG{g+gp}{\PYGZgt{}\PYGZgt{}\PYGZgt{} }\PYG{n}{ex} \PYG{o}{=} \PYG{n}{focus\PYGZus{}field\PYGZus{}beam\PYGZus{}plane}\PYG{p}{(}\PYG{p}{(}\PYG{l+m+mi}{256}\PYG{p}{,}\PYG{l+m+mi}{256}\PYG{p}{)}\PYG{p}{,} \PYG{p}{(}\PYG{l+m+mf}{0.1}\PYG{p}{,}\PYG{l+m+mf}{0.1}\PYG{p}{)}\PYG{p}{,} \PYG{n}{z} \PYG{o}{=} \PYG{l+m+mi}{4}\PYG{p}{,} \PYG{n}{lam}\PYG{o}{=}\PYG{o}{.}\PYG{l+m+mi}{5}\PYG{p}{,} \PYG{n}{NA} \PYG{o}{=} \PYG{p}{(}\PYG{o}{.}\PYG{l+m+mi}{4}\PYG{p}{,}\PYG{o}{.}\PYG{l+m+mi}{5}\PYG{p}{)}\PYG{p}{)}
\end{Verbatim}


\strong{See also:}

\begin{description}
\item[{{\hyperref[beams:biobeam.focus_field_beam]{\emph{\code{biobeam.focus\_field\_beam()}}}}}] \leavevmode
the 3d function

\end{description}



\end{fulllineitems}



\section{Cylindrical Lens}
\label{beams:cylindrical-lens}
{\hspace*{\fill}\includegraphics{{pupil_cylinder}.png}\hspace*{\fill}}
\index{focus\_field\_cylindrical() (in module biobeam)}

\begin{fulllineitems}
\phantomsection\label{beams:biobeam.focus_field_cylindrical}\pysiglinewithargsret{\code{biobeam.}\bfcode{focus\_field\_cylindrical}}{\emph{shape}, \emph{units}, \emph{lam=0.5}, \emph{NA=0.3}, \emph{n0=1.0}, \emph{return\_all\_field=False}, \emph{n\_integration\_steps=100}}{}
calculates the focus field for a perfect, aberration free cylindrical lens after
x polarized illumination via the vectorial debye diffraction integral (see \footnote[2]{
Colin J. R. Sheppard: Cylindrical lenses—focusing and imaging: a review, Appl. Opt. 52, 538-545 (2013)
}).
The pupil function is given by the numerical aperture NA
\begin{quote}\begin{description}
\item[{Parameters}] \leavevmode\begin{itemize}
\item {} 
\textbf{\texttt{shape}} (\emph{\texttt{Nx,Ny,Nz}}) -- the shape of the geometry

\item {} 
\textbf{\texttt{units}} (\emph{\texttt{dx,dy,dz}}) -- the pixel sizes in microns

\item {} 
\textbf{\texttt{lam}} (\emph{\texttt{float}}) -- the wavelength of light used in microns

\item {} 
\textbf{\texttt{NA}} (\emph{\texttt{float}}) -- the numerical aperture of the lens

\item {} 
\textbf{\texttt{n0}} (\emph{\texttt{float}}) -- the refractive index of the medium

\item {} 
\textbf{\texttt{return\_all\_fields}} (\emph{\texttt{boolean}}) -- if True, returns u,ex,ey,ez where ex/ey/ez are the complex field components

\item {} 
\textbf{\texttt{n\_integration\_steps}} (\emph{\texttt{int}}) -- number of integration steps to perform

\item {} 
\textbf{\texttt{return\_all\_fields}} -- if True returns also the complex vectorial field components

\end{itemize}

\item[{Returns}] \leavevmode
\begin{itemize}
\item {} 
\textbf{u} (\emph{ndarray}) --
the intensity of the focus field

\item {} 
\textbf{(u,ex,ey,ez)} (\emph{list(ndarray)}) --
the intensity of the focus field and the complex field components (if return\_all\_fields is True)

\end{itemize}


\end{description}\end{quote}
\paragraph{Example}

\begin{Verbatim}[commandchars=\\\{\}]
\PYG{g+gp}{\PYGZgt{}\PYGZgt{}\PYGZgt{} }\PYG{n}{u}\PYG{p}{,} \PYG{n}{ex}\PYG{p}{,} \PYG{n}{ey}\PYG{p}{,} \PYG{n}{ez} \PYG{o}{=} \PYG{n}{focus\PYGZus{}field\PYGZus{}cylindrical}\PYG{p}{(}\PYG{p}{(}\PYG{l+m+mi}{128}\PYG{p}{,}\PYG{l+m+mi}{128}\PYG{p}{,}\PYG{l+m+mi}{128}\PYG{p}{)}\PYG{p}{,} \PYG{p}{(}\PYG{l+m+mf}{0.1}\PYG{p}{,}\PYG{l+m+mf}{0.1}\PYG{p}{,}\PYG{o}{.}\PYG{l+m+mi}{1}\PYG{p}{)}\PYG{p}{,} \PYG{n}{lam}\PYG{o}{=}\PYG{o}{.}\PYG{l+m+mi}{5}\PYG{p}{,} \PYG{n}{NA} \PYG{o}{=} \PYG{o}{.}\PYG{l+m+mi}{4}\PYG{p}{,} \PYG{n}{return\PYGZus{}all\PYGZus{}field}\PYG{o}{=}\PYG{n+nb+bp}{True}\PYG{p}{)}
\end{Verbatim}
\paragraph{References}

\end{fulllineitems}

\index{focus\_field\_cylindrical\_plane() (in module biobeam)}

\begin{fulllineitems}
\phantomsection\label{beams:biobeam.focus_field_cylindrical_plane}\pysiglinewithargsret{\code{biobeam.}\bfcode{focus\_field\_cylindrical\_plane}}{\emph{shape=(128}, \emph{128)}, \emph{units=(0.1}, \emph{0.1)}, \emph{z=0.0}, \emph{lam=0.5}, \emph{NA=0.6}, \emph{n0=1.0}, \emph{ex\_g=None}, \emph{n\_integration\_steps=200}}{}
calculates the complex 2d input field at position -z of a      perfect, aberration free optical system
\begin{quote}\begin{description}
\item[{Parameters}] \leavevmode\begin{itemize}
\item {} 
\textbf{\texttt{shape}} (\emph{\texttt{Nx,Ny}}) -- the 2d shape of the geometry

\item {} 
\textbf{\texttt{units}} (\emph{\texttt{dx,dy}}) -- the pixel sizes in microns

\item {} 
\textbf{\texttt{z}} (\emph{\texttt{float}}) -- defocus position in microns, such that the beam would focus at z
e.g. an input field with z = 10. would hav its focus spot after 10 microns

\item {} 
\textbf{\texttt{lam}} (\emph{\texttt{float}}) -- the wavelength of light used in microns

\item {} 
\textbf{\texttt{NA}} (\emph{\texttt{float/list}}) -- the numerical aperture(s) of the illumination objective
that is either a single number (for gaussian beams) or an
even length list of NAs (for bessel beams), e.g.
NA = {[}0.5,0.55{]} lets light through the annulus 0.5\textless{}0.55 (making a bessel beam ) or
NA = {[}0.1,0.2,0.5,0.6{]} lets light through the annulus 0.1\textless{}0.2 and 0.5\textless{}0.6 making a
beating double bessel beam...

\item {} 
\textbf{\texttt{n0}} (\emph{\texttt{float}}) -- the refractive index of the medium

\item {} 
\textbf{\texttt{n\_integration\_steps}} (\emph{\texttt{int}}) -- number of integration steps to perform

\end{itemize}

\item[{Returns}] \leavevmode
\textbf{ex} --
the complex field

\item[{Return type}] \leavevmode
ndarray

\end{description}\end{quote}
\paragraph{Example}

\begin{Verbatim}[commandchars=\\\{\}]
\PYG{g+gp}{\PYGZgt{}\PYGZgt{}\PYGZgt{} }\PYG{c+c1}{\PYGZsh{} the input pattern of a bessel beam that will focus after 4 microns}
\PYG{g+gp}{\PYGZgt{}\PYGZgt{}\PYGZgt{} }\PYG{n}{ex} \PYG{o}{=} \PYG{n}{focus\PYGZus{}field\PYGZus{}cylindrical\PYGZus{}plane}\PYG{p}{(}\PYG{p}{(}\PYG{l+m+mi}{256}\PYG{p}{,}\PYG{l+m+mi}{256}\PYG{p}{)}\PYG{p}{,} \PYG{p}{(}\PYG{l+m+mf}{0.1}\PYG{p}{,}\PYG{l+m+mf}{0.1}\PYG{p}{)}\PYG{p}{,} \PYG{n}{z} \PYG{o}{=} \PYG{l+m+mf}{4.}\PYG{p}{,} \PYG{n}{lam}\PYG{o}{=}\PYG{o}{.}\PYG{l+m+mi}{5}\PYG{p}{,} \PYG{n}{NA} \PYG{o}{=} \PYG{p}{(}\PYG{o}{.}\PYG{l+m+mi}{4}\PYG{p}{,}\PYG{o}{.}\PYG{l+m+mi}{5}\PYG{p}{)}\PYG{p}{)}
\end{Verbatim}


\strong{See also:}

\begin{description}
\item[{{\hyperref[beams:biobeam.focus_field_cylindrical]{\emph{\code{biobeam.focus\_field\_cylindrical()}}}}}] \leavevmode
the 3d function

\end{description}



\end{fulllineitems}



\section{Bessel Lattices}
\label{beams:bessel-lattices}
{\hspace*{\fill}\includegraphics{{pupil_lattice}.png}\hspace*{\fill}}
\index{focus\_field\_lattice() (in module biobeam)}

\begin{fulllineitems}
\phantomsection\label{beams:biobeam.focus_field_lattice}\pysiglinewithargsret{\code{biobeam.}\bfcode{focus\_field\_lattice}}{\emph{shape}, \emph{units}, \emph{lam=0.5}, \emph{NA1=0.4}, \emph{NA2=0.5}, \emph{sigma=0.1}, \emph{kpoints=6}, \emph{n0=1.0}, \emph{n\_integration\_steps=100}}{}
calculates the focus field for a bessel lattice
The pupil function consists out of discrete points (kpoints) superimposed on an annulus (NA1\textless{}NA2)
which are smeared out by a 1d gaussian of given sigma creating an array of bessel beams in the
focal plane (see \footnote[3]{
Chen et al. Lattice light-sheet microscopy: imaging molecules to embryos at high spatiotemporal resolution. Science 346, (2014).
} ).
\begin{quote}\begin{description}
\item[{Parameters}] \leavevmode\begin{itemize}
\item {} 
\textbf{\texttt{shape}} (\emph{\texttt{Nx,Ny,Nz}}) -- the shape of the geometry

\item {} 
\textbf{\texttt{units}} (\emph{\texttt{dx,dy,dz}}) -- the pixel sizes in microns

\item {} 
\textbf{\texttt{lam}} (\emph{\texttt{float}}) -- the wavelength of light used in microns

\item {} 
\textbf{\texttt{NA1}} (\emph{\texttt{float/list}}) -- the numerical aperture of the inner ring

\item {} 
\textbf{\texttt{NA2}} (\emph{\texttt{float/list}}) -- the numerical aperture of the outer ring

\item {} 
\textbf{\texttt{sigma}} (\emph{\texttt{float}}) -- the standard deviation of the gaussian smear function applied to each point on the aperture
(the bigger sigma, the tighter the sheet in y)

\item {} 
\textbf{\texttt{kpoints}} (\emph{\texttt{int/ (2,N) array}}) -- defines the set of points on the aperture that create the lattice, can be
- a (2,N) ndarray, such that kpoints{[}:,i{]} are the coordinates of the ith point
- a single int, defining points on a regular polygon (e.g. 4 for a square lattice, 6 for a hex lattice)
\(k_i = \arcsin\frac{NA_1+NA_2}{2 n_0} \begin{pmatrix} \cos \phi_i \\ \sin \phi_i \end{pmatrix}\quad, \phi_i = \frac{\pi}{2}+\frac{2i}{N}\)

\item {} 
\textbf{\texttt{n0}} (\emph{\texttt{float}}) -- the refractive index of the medium

\item {} 
\textbf{\texttt{n\_integration\_steps}} (\emph{\texttt{int}}) -- number of integration steps to perform

\item {} 
\textbf{\texttt{return\_all\_fields}} (\emph{\texttt{boolean}}) -- if True, returns u,ex,ey,ez where ex/ey/ez are the complex vector field components

\end{itemize}

\item[{Returns}] \leavevmode
\begin{itemize}
\item {} 
\textbf{u} (\emph{ndarray}) --
the intensity of the focus field

\item {} 
\textbf{(u,ex,ey,ez)} (\emph{list(ndarray)}) --
the intensity of the focus field and the complex field components (if return\_all\_fields is True)

\end{itemize}


\end{description}\end{quote}
\paragraph{Example}

\begin{Verbatim}[commandchars=\\\{\}]
\PYG{g+gp}{\PYGZgt{}\PYGZgt{}\PYGZgt{} }\PYG{n}{u} \PYG{o}{=} \PYG{n}{focus\PYGZus{}field\PYGZus{}lattice}\PYG{p}{(}\PYG{p}{(}\PYG{l+m+mi}{128}\PYG{p}{,}\PYG{l+m+mi}{128}\PYG{p}{,}\PYG{l+m+mi}{128}\PYG{p}{)}\PYG{p}{,} \PYG{p}{(}\PYG{l+m+mf}{0.1}\PYG{p}{,}\PYG{l+m+mf}{0.1}\PYG{p}{,}\PYG{o}{.}\PYG{l+m+mi}{1}\PYG{p}{)}\PYG{p}{,} \PYG{n}{lam}\PYG{o}{=}\PYG{o}{.}\PYG{l+m+mi}{5}\PYG{p}{,} \PYG{n}{NA1} \PYG{o}{=} \PYG{o}{.}\PYG{l+m+mi}{44}\PYG{p}{,} \PYG{n}{NA2} \PYG{o}{=} \PYG{o}{.}\PYG{l+m+mi}{55}\PYG{p}{,} \PYG{n}{kpoints} \PYG{o}{=} \PYG{l+m+mi}{6}\PYG{p}{)}
\end{Verbatim}
\paragraph{References}

\end{fulllineitems}

\index{focus\_field\_lattice\_plane() (in module biobeam)}

\begin{fulllineitems}
\phantomsection\label{beams:biobeam.focus_field_lattice_plane}\pysiglinewithargsret{\code{biobeam.}\bfcode{focus\_field\_lattice\_plane}}{\emph{shape=(256}, \emph{256)}, \emph{units=(0.1}, \emph{0.1)}, \emph{z=0.0}, \emph{lam=0.5}, \emph{NA1=0.4}, \emph{NA2=0.5}, \emph{sigma=0.1}, \emph{Npoly=6}, \emph{n0=1.0}, \emph{apodization\_bound=10}, \emph{ex\_g=None}, \emph{n\_integration\_steps=100}}{}
\end{fulllineitems}



\chapter{Examples}
\label{examples::doc}\label{examples:examples}

\section{Plane wave scattered by sphere}
\label{examples:plane-wave-scattered-by-sphere}
\begin{Verbatim}[commandchars=\\\{\}]
\PYG{c+c1}{\PYGZsh{} create the refractive index difference}
\PYG{n}{x} \PYG{o}{=} \PYG{l+m+mf}{0.1} \PYG{o}{*} \PYG{n}{np}\PYG{o}{.}\PYG{n}{arange}\PYG{p}{(}\PYG{o}{\PYGZhy{}}\PYG{l+m+mi}{128}\PYG{p}{,}\PYG{l+m+mi}{128}\PYG{p}{)}
\PYG{n}{Z}\PYG{p}{,} \PYG{n}{Y}\PYG{p}{,} \PYG{n}{X} \PYG{o}{=} \PYG{n}{np}\PYG{o}{.}\PYG{n}{meshgrid}\PYG{p}{(}\PYG{n}{x}\PYG{p}{,}\PYG{n}{x}\PYG{p}{,}\PYG{n}{x}\PYG{p}{,}\PYG{n}{indexing} \PYG{o}{=} \PYG{l+s+s2}{\PYGZdq{}}\PYG{l+s+s2}{ij}\PYG{l+s+s2}{\PYGZdq{}}\PYG{p}{)}
\PYG{n}{R} \PYG{o}{=} \PYG{n}{np}\PYG{o}{.}\PYG{n}{sqrt}\PYG{p}{(}\PYG{n}{X}\PYG{o}{*}\PYG{o}{*}\PYG{l+m+mi}{1}\PYG{o}{+}\PYG{n}{Y}\PYG{o}{*}\PYG{o}{*}\PYG{l+m+mi}{2}\PYG{o}{+}\PYG{n}{Z}\PYG{o}{*}\PYG{o}{*}\PYG{l+m+mi}{2}\PYG{p}{)}
\PYG{n}{dn} \PYG{o}{=} \PYG{l+m+mf}{0.05}\PYG{o}{*}\PYG{p}{(}\PYG{n}{R}\PYG{o}{\PYGZlt{}}\PYG{l+m+mf}{2.}\PYG{p}{)}

\PYG{c+c1}{\PYGZsh{} create the computational geometry}
\PYG{n}{m} \PYG{o}{=} \PYG{n}{Bpm3d}\PYG{p}{(}\PYG{n}{dn} \PYG{o}{=} \PYG{n}{dn}\PYG{p}{,} \PYG{n}{units} \PYG{o}{=} \PYG{p}{(}\PYG{l+m+mf}{0.1}\PYG{p}{,}\PYG{l+m+mf}{0.1}\PYG{p}{,}\PYG{l+m+mf}{0.1}\PYG{p}{)}\PYG{p}{,} \PYG{n}{lam} \PYG{o}{=} \PYG{l+m+mf}{0.5}\PYG{p}{)}

\PYG{c+c1}{\PYGZsh{} propagate a plane wave and return the intensity}
\PYG{n}{u} \PYG{o}{=} \PYG{n}{m}\PYG{o}{.}\PYG{n}{\PYGZus{}propagate}\PYG{p}{(}\PYG{p}{)}

\PYG{c+c1}{\PYGZsh{} vizualize}
\PYG{k+kn}{import} \PYG{n+nn}{matplotlib.pyplot} \PYG{k+kn}{as} \PYG{n+nn}{plt}
\PYG{n}{plt}\PYG{o}{.}\PYG{n}{subplot}\PYG{p}{(}\PYG{l+m+mi}{1}\PYG{p}{,}\PYG{l+m+mi}{2}\PYG{p}{,}\PYG{l+m+mi}{1}\PYG{p}{)}
\PYG{n}{plt}\PYG{o}{.}\PYG{n}{imshow}\PYG{p}{(}\PYG{n}{u}\PYG{p}{[}\PYG{o}{.}\PYG{o}{.}\PYG{o}{.}\PYG{p}{,}\PYG{l+m+mi}{128}\PYG{p}{]}\PYG{p}{,} \PYG{n}{cmap} \PYG{o}{=} \PYG{l+s+s2}{\PYGZdq{}}\PYG{l+s+s2}{hot}\PYG{l+s+s2}{\PYGZdq{}}\PYG{p}{)}
\PYG{n}{plt}\PYG{o}{.}\PYG{n}{title}\PYG{p}{(}\PYG{l+s+s2}{\PYGZdq{}}\PYG{l+s+s2}{zy slice}\PYG{l+s+s2}{\PYGZdq{}}\PYG{p}{)}
\PYG{n}{plt}\PYG{o}{.}\PYG{n}{subplot}\PYG{p}{(}\PYG{l+m+mi}{1}\PYG{p}{,}\PYG{l+m+mi}{2}\PYG{p}{,}\PYG{l+m+mi}{2}\PYG{p}{)}
\PYG{n}{plt}\PYG{o}{.}\PYG{n}{imshow}\PYG{p}{(}\PYG{n}{u}\PYG{p}{[}\PYG{l+m+mi}{128}\PYG{p}{,}\PYG{o}{.}\PYG{o}{.}\PYG{o}{.}\PYG{p}{]}\PYG{p}{,} \PYG{n}{cmap} \PYG{o}{=} \PYG{l+s+s2}{\PYGZdq{}}\PYG{l+s+s2}{hot}\PYG{l+s+s2}{\PYGZdq{}}\PYG{p}{)}
\PYG{n}{plt}\PYG{o}{.}\PYG{n}{title}\PYG{p}{(}\PYG{l+s+s2}{\PYGZdq{}}\PYG{l+s+s2}{xy slice}\PYG{l+s+s2}{\PYGZdq{}}\PYG{p}{)}
\end{Verbatim}


\section{Light sheet through cell phantom}
\label{examples:light-sheet-through-cell-phantom}

\section{Computing the psf inside a cell phantom}
\label{examples:computing-the-psf-inside-a-cell-phantom}

\section{Aberration from sphere}
\label{examples:aberration-from-sphere}

\chapter{Some Examples}
\label{examples:some-examples}
or not?

\begin{Verbatim}[commandchars=\\\{\}]
\PYG{k}{print} \PYG{l+s+s2}{\PYGZdq{}}\PYG{l+s+s2}{huhu}\PYG{l+s+s2}{\PYGZdq{}}
\end{Verbatim}

\begin{Verbatim}[commandchars=\\\{\}]
\PYG{n}{huhu}
\end{Verbatim}



\renewcommand{\indexname}{Index}
\printindex
\end{document}
